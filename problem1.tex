
%Obs! Macros are used in order to simplify typing. While this is usually bad practice when others are meant to read your code,
%this latex code was only meant to be read by two people agreeing to use macros in order to save time and simplify.

\section*{Problem 1 - Attitude Control of Satellite}
\todo[inline]{Write in an overall understanding of the assignment problem here. What are we doing in this assignment? What are we achieving? What are we learning? What is it the professor and student assistants wants us to have understood from all of this?}
\listoftodos[List of things that need to be done before turn in:]

\subsection*{Problem 1.1} 
Equation (1) from the assignment is:
\begin{equation}
\label{eq:dynamics}
	\begin{aligned}
		\dot{\mathbf{q}} = \mathbf{T}_q (\mathbf{q} ) \boldsymbol{\omega} \\
		\mathbf{I}_{CG} \dot{\boldsymbol{\omega}} - \mathbf{S} (\mathbf{I}_{CG} \boldsymbol{\omega} ) \boldsymbol{\omega} & =  \boldsymbol{\tau}
	\end{aligned}	
\end{equation}
We find the equilibrium point of  $\bb{q}$ \eqref{eq:dynamics} where the label makes sure that the correct equation number is used. If you want to write an equation directly in the text (outside of the equation environment), use: $\dot{\mathbf{q}} = \mathbf{T}_q (\mathbf{q} ) \boldsymbol{\omega}$. % You have to use the dollar sign to write math symbols within a text.

A matrix (and an equation without equation number) can be created as: 
\begin{equation*}	% The star indicates that you don't want to give this equation a number. Normally used if you don't refer to the equation.
	\mathbf{A} = 
	\begin{bmatrix}
		a & b & c \\ d & e & f \\ g & h & i
	\end{bmatrix}
\end{equation*}

\subsection*{Problem 1.2}
Answer Problem 1.2 here. Bold words can be written as \textbf{something bold}. It is also possible to create a new section level:
\subsubsection*{Inner Section 1}
\emph{text..}

\subsubsection*{Inner Section 2}
...

\subsection*{Problem 1.3}
Answer Problem 1.3 here. Equation (2) from the assignment can be written as: 
\begin{equation}
  \label{eq:tau}
  \mathbf{\tau} = -\mathbf{K}_d \boldsymbol{\omega} - k_p \boldsymbol{\epsilon}
\end{equation}

\subsection*{Problem 1.4}
The quaternion error can be written as
 \begin{equation}
	 \tilde{\mathbf{q}} := \left[
	 \begin{array}{c}
		 \tilde{\eta} \\
		 \tilde{\epsilon}
	 \end{array}
	 \right] = \bar{\mathbf{q}}_d \otimes \mathbf{q} 
 \end{equation}

\subsection*{Problem 1.5}
...

\subsection*{Problem 1.6}
...

\subsection*{Problem 1.7}
The Lyapunov function can be written as 
 \begin{equation}
	 V = \frac{1}{2} \tilde{\boldsymbol{\omega}}^{\top} \mathbf{I}_{CG}\tilde{\boldsymbol{\omega}} + 2 k_p (1-\tilde{\eta})
 \end{equation}

\subsection*{Problem 1.8}
...

% Note that \mathbf can be used for bold letters in math mode (within equations and dollar signs). \boldsymbol can be used to get bold greek letters.  